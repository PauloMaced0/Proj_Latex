\input{packages.tex} 

\begin{document}	

	%Definições do Relatório

%Dados Gerais:

\def\titulo{Spotify}

\def\data{19 de Dezembro de 2021}

\def\departamento{Departamento de Electrónica Telecomunicações e Informática}

\def\versao{Ver.:1.0}

\def\empresa{Universidade de Aveiro}

\def\logotipo{logotipoua.png}

%Dados dos Autores:

%primeiro autor:

\def\autorluis{Luís Ernesto Castro Leal} 
\def\numautorluis{Nº Mec.: 103511}
\def\contactoautorluis{lecl@ua.pt}

%segundo autor:

\def\autorpaulo{Paulo Sérgio Ribeiro Macedo} 
\def\numautorpaulo{Nº Mec.: 102620}
\def\contactoautorpaulo{paulomacedo@ua.pt}

\def\autores{\autorluis \\ \autorpaulo}


	\input{capa.tex}

	%Página de Título:

\predate{\begin{flushright}\small}

\postdate{\par\end{flushright}}

\title{ 
	{\huge\textbf{\titulo} } \\  
	{\large \departamento\\ \empresa} 
}

\author{
		\autorluis , \numautorluis \\
		\contactoautorluis 
		\and
		\autorpaulo , \numautorpaulo \\
		\contactoautorpaulo				}

\date{\vspace{\fill}{\data}}

\maketitle

	\begin{abstract}

Neste trabalho exploramos um serviço de streaming, o Spotify. Como este se tem tornado um dos principais serviços para a indústria músical, é pertinente elaborar um conjunto de informações e curiosidades, não só sobre o serviço mas também daqueles que apresentam o seu conteúdo disponível para reprodução.
É mencionado o conceito de streaming, que não é conhecido por muitas pessoas, sendo este um conceito importante, visto que é a base do funcionamento deste serviço.
Uma vez que, o Spotify não é a única plataforma disponível para este tipo de serviço, existe um enorme pressão por parte da concorrência, para que continue a enovar e apresentar novas funcionalidades, sempre com o medo de que possa ser ultrapassada e entrar em decaimento.
No caso do Spotify, a quantidade de opções que tem para usufruir do serviço é relativamente alargada, tendo em conta outros serviços, o Spotify dá a opção de escolher o serviço grátis ou então o pago(premium), tendo o serviço pago várias opções por onde escolher,apresentado planos no caso de tencionar usufruir de um serviço individual ou conjunto.\\ 
Por fim, o que seria desta plataforma sem os seus artistas e os utilizadores para propocionar, não só fama á empresa, mas como também receitas que permitem estabelecer novos contratos com produtoras e assim garantir um serviço completo e de excelência. Portanto, é relevante apresentar a porção dos utilizadores que prefer elevar a sua experiência a outro nível e desfrutar do plano premium, e também os artistas com mais nome na plataforma e no mundo da música.

\end{abstract}
	

%ÍNDICE:

	\renewcommand{\contentsname}{Índice}
	\tableofcontents
	\listoffigures
	\listoftables
	\pagenumbering{roman}

%HEADERS & FOOTERS:

\pagestyle{fancy}
\fancyhf{}
\rhead{\titulo}
\lhead{Introdução}
\cfoot{\thepage}
\pagenumbering{arabic}

\chapter{Introdução}
\label{introduçao}

De forma a acompanhar a evolução tecnológica, a indústria musical viu-se obrigada a sair da sua "zona de conforto" e a procurar vender e publicitar os seus artistas de outras formas além dos formatos convencionais (vinil, cassete, cd, etc.).\\
	Uma das primeiras plataformas a fazer esta transição do formato físico para o digital foi o Napster, sendo que, atualmente as três principais plataformas de \textit{streaming}, na Europa, são o Spotify, o Apple Music e o SoundCloud.\\
	Neste relatório é abordado o serviço Spotify, uma vez que é, indubitávelmente, o serviço de \textit{streaming} que mais receitas produz, que mais utilizadores possui e que mais revolucionou a indústria musical.\\
	Disponível para as mais variadas plataformas, o Spotify regista, até à data, um total de 381 milhões de utilizadores, fazendo dele a maior plataforma de \textit{streaming} do mundo. Com um catálogo completamente licenciado, a plataforma disponibiliza mais de 70 milhões de músicas e quase 3 milhões de \textit{podcasts}, fornecendo ainda funcionalidades tais como a criação de \textit{playlists} geradas automaticamente pela plataforma com base nas preferências do utilizador e o registo através de uma conta Facebook, que permite ao utilizador acompanhar os gostos musicais dos seus seguidores.\\

Este documento está dividido em 4 capítulos.
Depois desta introdução, no \autoref{logo} \ é apresentado o logótipo atual do Spotify e as suas características, de seguida, no \autoref{streaming}, é dada a conhecer uma breve explicação sobre o que é o \textit{streaming}, de modo a introduzir este novo conceito relacionado com as mais recentes formas de ouvir música, sendo apresentadas nos seguintes capítulos, \autoref{plataforma} e \autoref{estaticadados}, informações sobre as opções a que um utilizador poderá ter caso queira aderir ao serviço \textit{premium} e também o respetivo valor cobrado, o tipo de conteúdo que pode esperar encontrar e também alguns dados sobre os artistas/\textit{podcasts} mais ouvidos. 
Finalmente, na \autoref{conclusao} é feita uma apreciação geral do trabalho e é feto um balanço positivo do que o serviço dá à comunidade.
 
\chapter{Spotify - Logótipo}
\label{logo}
O logótipo icónico do Spotify, é facilmente reconhecível e legível pelas gerações mais recentes. É minimalista e moderno porque tem menos elementos gráficos do que cor, forma e tipo de letra, realçando também o equilíbrio do esquema de cores.

Todos os logótipos do Spotify apresentam ondas sonoras. Três linhas representam energia ou ondas musicais, desde o logótipo original até ao atual. Esta forma é o que faz do Spotify uma empresa musical. Estas linhas horizontais, que são livres para se mover, transmitem uma sensação de distância, calma e infinidade.
Logótipos fáceis de lembrar são os melhores. Os logótipos mais notáveis são envolventes, divertidos e intrigantes, como o do caso do Spotify, que é verdadeiramente apelativo, não podem ser ignorado, perdido ou esquecido na multidão.

O Spotify transmite a sua criatividade de design não só no seu logótipo, mas também na sua aplicação e no seu \textit{website}.\\
\\
\\

	\begin{figure}[h!]

		\centering

		\includegraphics[width=4cm]{spotifylogo.png}

		\caption{Logótipo Atual do Spotify}

		\label{fig.logotipo}

	\end{figure} 

\chapter{Streaming}
\label{streaming}
\section{O Conceito}
Utilizado primordialmente pela empresa Data Eletronics Inc. para designar uma tecnologia relacionada com cassetes e atribuído, no início da década de 90, como  descrição para o serviço VOD (Video on Demand), o termo \textit{streaming} representa, de forma geral, um método através do qual é possível aceder a conteúdos multimédia sem necessitar de os descarregar, requerindo, apenas, uma conexão à Internet. Em suma, \textit{streaming} é um método de distribuição de conteúdo que é entregue e consumido de forma contínua.
 Cada plataforma de \textit{streaming}, possui, em norma, uma área de especialização. Como são os casos do Spotify, para a música, da Netflix, filmes e séries e do Stadia, vídeo jogos. 

\section{Serviços Concorrentes}
Hoje em dia, a escolha feita por cada utilizador para eleger o serviço de \textit{streaming} ideal é baseada em características, tais como: a finalidade do serviço de \textit{streaming}, se existem opções gratuitas disponíveis, a presença de anúncios, o custo da assinatura e o tamanho da biblioteca de música disponibilizada. Desta forma, tendo em conta o largo número de opções existentes, o Spotify apresenta uma grande competição, sendo exemplos disso os seguintes serviços:

	\begin{itemize}
	\item Deezer
	\item Napster
	\item Tidal
	\item Google Music
 	\item Youtube Music
	\item Apple Music
	\item Qobuz	
	\item Pandora Radio
	\item iHeart Radio
	\end{itemize}

No caso específico do Pandora, dispõe de uma lista extensa de estações de rádio, cujo propósito é ajudar os utilizadores a descobrir novos artistas e novas músicas. Dentro do género de serviços que estão destinados a apresentar novos artistas e músicas, temos também o iHeart Radio, que apresenta igualmente uma ampla gama de estações de rádio online. O iHeart Radio oferece unicamente serviços gratuitos, porém, só permite 15 \textit{skips} combinados por dia e 6 \textit{skips} por estação de rádio por hora.
No entanto, apesar de alguns serviços possuírem melhores funcionalidades e planos com uma melhor relação preço/qualidade, a maioria dos utilizadores acaba por optar pelo Spotify como o seu serviço de eleição, graças ao seu extenso catálogo licenciado e às funcionalidades de interação entre utilizadores de que este dispõe.   



\chapter{Spotify - A Plataforma}
\label{plataforma}
\section{Apresentação}
O Spotify foi fundado a 23 de abril de 2006 por Daniel Ek e Martin Lorentzon, em Estocolmo, na Suécia, no entanto o serviço de \textit{streaming} só foi lançado oficialmente a 7 outubro de 2008, estreando-se (apenas por convite) na Escandinávia, Reino Unido, França e Espanha. Países como a Holanda e os Estados Unidos, que foram os seguintes a prever o potencial sucesso desta nova plataforma, só tiveram acesso em 2009/2010, respetivamente. Portugal, só teve esta plataforma nas suas mãos 5 anos depois do seu lançamento, sendo um dos últimos países da união europeia a ter acesso. \\
 Quanto a Daniel Ek, este é um empreendedor sueco na área da tecnologia e a sua jornada na área começou aos 14 anos, quando começou a desenvolver \textit{websites} para pequenos negócios. De seguida, edificou os seus próprios servidores, propocionando serviços de hospedagem na internet. Após ter concluído os seus estudos universitários, fundou a empresa de anúncios online Advertigo, que acabou por vender, em 2006, à empresa sueca Tradedoubler. Chegou também a ocupar posições administrativas nas empresas Jajja Communications, Stardoll e uTorrent, tudo isto antes de fundar o Spotify com Martin Lorentzon.
 Já Martin Lorentzon, cofundador da Tradedoubler e do Spotify, viu a sua carreira profissional ascender, quando em setembro de 1999, juntamente com Felix Hagno, fundou a Netstrategy, que mais tarde se transformou na Tradedoubler. Esta última, como referido anteriormente, comprou a Advertigo, empresa de Daniel Ek. Esta junção das duas empresas levou a que se tornassem amigos com projetos em comum, dando assim origem ao Spotify.\\
 Para que o vasto catálogo do Spotify fosse disponibilizado para os seus utilizadores, a empresa viu-se obrigada a assinar acordos com produtoras musicais tais como a \ac{umg},\ac{wmg}, \ac{smeg}, tendo esta 5~\% das ações da empresa. Em 2018 foram lançados dados sobre os valores que a empresa pagou, desde 2006, a artistas e produtoras, rondando este valor os 9.7 mil milhões €.\\
 Quanto à sua disponibilidade a nível territorial, o Spotify está presente em 178 países, tendo acrescentado 81 à lista apenas no último ano.\\
 De acordo com dados emitidos pela própria empresa, até à primeira metade de 2021, o Spotify emprega 7085 funcionários, revelando ainda que o seu salário anual médio rondava os 122.814 euros.

   

\section{Funcionamento e Adesão}
Os planos de adesão do Spotify dividem-se em duas categorias principais, o Spotify Free (gratuito) e o Spotify Premium (pago).
	Enquanto que o plano gratuito apenas permite ouvir músicas no modo \textit{shuffle} e com pausas para anúncios, o pago dá a possibilidade de fazer \textit{download} das músicas para ouvir \textit{offline}; navegar nas playlist sem restrições, podendo-se selecionar a música que se pretende reproduzir; e ter acesso a áudio de qualidade superior, equivalente a 320 kbit/s.\\
	
As possibilidades dentro dos planos Premium são:
	\begin{itemize}
	\item Individual – 6,99€ (uma conta)
	\item Estudante – 3,49€ (uma conta)
	\item Duo – 8,99€ (duas contas)
	\item Família – 11,99€ (até seis contas).
	\end{itemize}
			

\chapter{Estatísticas e Dados}
\label{estaticadados}
\section{Conteúdo Disponível}
\subsection{Música}

Seria impossível realizar um relatório sobre o Spotify se referir o óbvio, que é o catálogo musical que tem para oferecer. Do \textit{pop} ao \textit{rock} e do clássico ao eletrónico, são muito raros e quase impossíveis de detetar os casos em que a plataforma não possui o conteúdo que procuramos, até porque, de momento, o seu catálogo é composto por mais de 70 milhões de faixas. A par do catálogo extenso, o Spotify acaba também por oferecer uma espécie de serviço personalizado. Graças à recolha constante de dados, o algoritmo deteta padrões relativos aos gostos musicais do utilizador e sugere-lhe músicas dentro dos mesmos géneros a que este está acostumado, facilitando a descoberta por novas músicas e ajudando artistas menos conhecidos a obterem reconhecimento.\\ 
O impacto do Spotify na indústria musical é notório de tal forma que, até à data, a música que mais vezes foi reproduzida é a “Shape of You” do Ed Sheeran, que conta com 2.995 mil milhões de reproduções, um número que outrora seria impossível de atingir e que, por si só, reflete o número de utilizadores e a capacidade de exposição que o serviço tem.\\
A popularidade das músicas nem sempre se deve ao artista, facto que é possível comprovar pela fama que as faixas ganham com exposição dada por figuras públicas nas mais variadas redes sociais, mas além deste fator, também importa salientar a publicidade feita pelo Spotify que, de forma semanal, dá espaço a novas músicas nas suas \textit{playlists}, que possuem milhões de seguidores.\\
Um dos exemplos mais relevantes será, certamente, o de Billie Eilish, que com apenas 20 anos é, desde já, a quinta artista com mais seguidores no Spotify, contabilizando 53.86 milhões de fãs no serviço de \textit{streaming}. Além deste feito, conta ainda com 3 faixas cujo número de reproduções ultrapassa os mil milhões.


\begin{table}[H]
	\centering
	\caption{Músicas mais reproduzidas de todos os tempos}
	\begin{tabular}{l|l|l|l}\hline
	Posição & Música & Artista & Nº de Reproduções \\
	\hline
	1       &    Shape of You    &   Ed Sheeran   &    2.995M        \\
	2       &    Blinding Lights    & The Weeknd  & 2.688M   \\
	3       &     Dance Monkey   &  Tones and I       &  2.430M   \\
	4       &   Rockstar  &   Post Malone    &      2.315M             \\
	5       &   Someone You Loved     &   Lewis Capaldi      &  2.188M      
\end{tabular}
\end{table}
\subsubsection{Géneros Musicais e Playlists}

A música não é toda igual e, como tal, divide-se em géneros. É seguro dizer-se que ao longo dos anos, a "moda" vai mudando e, tendo já ocupado o "trono" a Ópera, o \textit{Jazz}, o \textit{Pop}, o \textit{Rock}, entre outros. Em 2017, registou-se algo inédito, pois o \textit{Rock} foi finalmente ultrapassado e, desde então, tem-se vindo a verificar um claro domínio por parte do \textit{Hip-hop/R \& B} nas plataformas de \textit{streaming}.\\
Este mais recente domínio do \textit{Hip-hop/R \& B} deve-se, acima de tudo, à influência que tem tido na música \textit{Pop}, o que lhe acaba por dar A visibilidade que não possuía anteriormente. Importa referir que a influência transmitida pelo \textit{Hip-hop/R \& B} ao \textit{Pop} provém de parcerias entre grandes nomes de ambos os géneros musicais.\\
Quanto a \textit{playlists}, o serviço disponibiliza uns impressionantes 4 mil milhões de opções e, 50\% das 20 \textit{playlists} com mais seguidores são compostas por música \textit{Pop}.


\subsection{Podcasts}

A seguir à música, os \textit{podcasts} são o conteúdo mais “consumido” pelos utilizadores do Spotify e tal como a música, também estes estão divididos em géneros.\\
Numa sociedade em que o tempo parece cada vez mais escasso e a disponibilidade e interesse da população em ler um livro aparentam ser reduzidas, os \textit{podcasts} tornaram-se uma excelente forma de obtenção de informação, uma vez que o consumidor poderá estar a realizar outras tarefas enquanto ouve algo sobre um assunto à sua escolha. Desta forma, importa referir dois dos \textit{podcasts} mais ouvidos do presente ano, uma vez que ambos partilham o facto de convidarem figuras bem sucedidas nas suas áreas para partilharem as suas ideias e experiências. Os exemplos referidos são o “The Joe Rogan Experience” e o “TED Talks Daily”.\\
O primeiro é apresentado por Joe Rogan, um comediante e comentador de \ac{ufc}. O conteúdo do \textit{podcast} varia consoante os convidados, uma vez que a cada episódio novo o moderador conversa com uma figura pública diferente e desde dezembro de 2020 que o podcast é exclusivo do Spotify, graças a um acordo assinado entre ambos que contempla uma cláusula de assinatura no valor 100 milhões de euros.\\
O segundo caso é o do “TED Talks Daily”, \textit{podcast} patrocinado pela organização sem fins lucrativos americana \ac{ted}, cujo slogan é “Ideas worth spreading” - ideais que valem a pena partilhar. Como tal, a temática dos episódios varia dependendo dos convidados, mas o objetivo passa sempre por dar tempo de antena a figuras bem sucedidas na sua área de trabalho para que possam educar e até mesmo motivar a sua audiência.\\
Além destes programas, que estão diretamente relacionados ao género "Sociedade e Cultura", o utilizador pode optar por muitos mais géneros, como por exemplo, "Notícias e Política", "Educação" e ainda "Negócios e Tecnologia".\\
Atualmente, o Spotify conta com aproximadamente 2.9 mil milhões de \textit{podcasts}, valor 5.8 vezes superior ao registado em 2019.

\begin{table}[H]
	\centering
	\caption{Podcast mais reproduzidos}
	\begin{tabular}{l|l}
	Posição & Podcast \\
	\hline
	1       &     The Joe Rogan Experience     \\
	2       &    Call Her Daddy \\
	3       &     Crime Junkie \\
	4       &        TED Talks Daily  \\
	5       &   The Daily
\end{tabular}
\end{table}
\section{Os Utilizadores}
Até à data, o Spotify conta com aproximadamente 381 milhões utilizadores e  cerca de 165 milhões são assinantes premium. A maior percentagem dos assinantes premium são europeus 69~\%, sendo os restantes da América do Norte (29~\%), América Latina (20~\%) e os últimos 11~\% estão distribuídos pelo resto do mundo.\\
Projeta-se que no ano de 2021, a empresa lucre entre 9.26 a 9.66 mil milhões de euros, valor 693 vezes superior ao lucro obtido em 2009 (13.26 milhões).
	

\begin{table}[H]

	\centering
	\caption{Distribuição dos serviços Premium}
	\begin{tabular}{l|l}
	Região & Serviços Premium \\
	\hline
	Europa      &    40\%  \\
	América do Norte       & 29\%  \\
	America Latina       &  20\%  \\
	Resto do Mundo       &    11\%
	\end{tabular}
\end{table}

\chapter{Conclusão}
\label{conclusao}

A sociedade atual vive rodeada de tecnologia, ao ponto de já estar tão mergulhada na transmissão de informação que muitas vezes nem se apercebe ou lembra da infinidade de conteúdos que ouviu recentemente.\\
O \textit{streaming} foi uma tecnologia que propocionou a difusão de conteúdos multimédia tanto de aúdio, como de vídeo, através da internet, sem ser necessário recorrer ao \textit{download} do conteúdo propriamente dito (por vezes ilegal).\\
Esta técnica tem diversas vantagens, sendo algumas delas a reprodução de conteúdos protegidos por direitos de autor e a economização de espaço no disco rígido dos equipamentos utilizados para a visualização dos
conteúdos, sendo uma das grandes vantagens de uma plataforma como o Spotify.\\
É notório que, ao longo dos últimos anos, a indústria musical tem vindo a sofrer grandes mudanças, visto que os consumidores deixaram de comprar um álbum físico, ou até mesmo uma faixa em formato de \ac{mp3}, para passarem a adquirir um modelo novo de assinatura que lhes permite ter acesso ao mais variado e extenso leque de conteúdos musicais.\\
Estas mudanças são, em grande medida, resultado não só dos avanços tecnológicos, mas também de alterações económicas e ainda culturais. O aparecimento do \textit{streaming} permitiu que os seus utilizadores pudessem ter acesso a inúmeras possibilidades de conteúdos digitais que, de outra forma, estariam impedidos de adquirir e de descobrir.\\
Assim, com o aparecimento deste novo serviço, exploraram-se novos modelos de negócio, surgindo novas práticas de circulação musical e o seu respetivo consumo, criando-se assim um novo mercado para a indústria musical.\\
Em contrapartida, o começo não foi fácil. Embora o \textit{streaming} musical estivesse a crescer, existia e ainda existe uma forte aposta nos \textit{downloads} ilegais. Face à grande queda notada nas vendas de \ac{cd}’s, um novo modelo de serviço musical teria ainda de desenvolver uma relação de confiança com os consumidores, que ainda hesitavam em adotar esta tecnologia como o seu principal método de consumo.\\
Os utilizadores reconhecem os riscos da pirataria de conteúdo como indiferentes ou quase nulos para afetar os seus comportamentos atuais, sendo que a perceção de baixo risco face à ilegalidade das suas ações torna quase impossível umas alteração nos seus hábitos, dificultanto assim o combate à pirataria. As tendências de pirataria digital surgem como resultado de vários fatores, tais como a indisponibilidade para pagar por uma alternativa legal e o baixo risco percepcionado por parte do consumidor face a estas atitudes ilegais. E desta forma, o Spotify vem como uma solução para resolver este problema, dando a opção aos utilizadores de usufruir do serviço de forma gratuita ou paga, sem que os admiradores de música tenham que recorrer a métodos ilegais para disfrutarem da mesma.\\
O Spotify é também uma fantástica ferramenta para novos artistas, uma vez que dá uma grande exposição ao seu trabalho (através das \textit{playlists}), conseguindo assim crescer como artista e ao mesmo tempo ser remunerado, dependendo, obviamente, do número de reproduções que alcança.\\  
Em suma, a indústria musical tem vindo a sofrer grandes transformações, que são, na sua maioria, derivadas dos avanços tecnológicos. A Internet e a partilha ilegal de ficheiros digitais de fácil acesso são um problema constante, colocando em causa a rentabilidade e a viabilidade desta
indústria, mas, a pouco e pouco, os serviços de \textit{streaming} e de venda conteúdo trabalham de forma a tornar cada vez mais difícil a sua aquisição ilegal. O \textit{streaming} de música acaba por surgir como uma resposta às alterações ressentidas no mercado digital e, como tal, tem vindo a ser muito bem sucedido na tarefa que desempenha, que é, principalmente, alimentar a indústria e fazê-la crescer. Desto modo, é inegável afirmar que o Spotify é um dos principais pilares da nova era de conteúdos digitais e o seu modelo veio para ficar.

\chapter*{Acrónimos}

\begin{acronym}

	\acro{ua}[UA]{Universidade de Aveiro}
	\acro{ll}[LL]{Autor: \autorluis , \numautorluis }
	\acro{pm}[PM]{Autor: \autorpaulo , \numautorpaulo }
	\acro{umg}[UMG]{Universal Music Group}
	\acro{smeg}[SMEG]{Sony Music Entertainment Group}
	\acro{wmg}[WMG]{Warner Music Group}
	\acro{cd}[CD]{Compact Disc}
	\acro{mp3}[MP3]{MPEG-1 Audio Layer III}
	\acro{ufc}[UFC]{Ultimate Fighting Championship}
	\acro{ted}[TED]{Technology, Entertainment, Design}
	
\end{acronym}
\chapter*{Contribuições dos Autores}

Projeto elaborado por Paulo Macedo e Luís Leal, estudantes de engenharia de computadores e informática da Universidade de Aveiro.



\begin{thebibliography}{9}

\bibitem{wikipedia}
\textit{https://pt.wikipedia.org/wiki/Spotify, acedido 18/12/2021.} 

\bibitem{wikipedia}
\textit{https://pt.wikipedia.org/wiki/Daniel\textunderscore Ek, acedido 18/12/2021.}

\bibitem{wikipedia}
\textit{https://en.wikipedia.org/wiki/Martin\textunderscore Lorentzon, acedido 18/12/2021.}

\bibitem{spotify}
\textit{https://www.spotify.com/pt-en/, acedido 18/12/2021.}

\bibitem{newroom}
\textit{https://newsroom.spotify.com/company-info/, acedido 18/12/2021.}

\bibitem{investors}
\textit{https://investors.spotify.com/home/default.aspx, acedido 18/12/2021.}

\bibitem{backinko}
\textit{https://backlinko.com/spotify-users, acedido 18/12/2021.}

\bibitem{podcast}
\textit{https://www.podcastinsights.com/podcast-statistics/, acedido 18/12/2021.}

\bibitem{wikipediasongs}
\textit{https://en.wikipedia.org/wiki/List\textunderscore of\textunderscore most-streamedsongs\textunderscore on\textunderscore Spotify, acedido 18/12/2021.}

\bibitem{logo}
\textit{https://turbologo.com/pt/blog/logotipo-spotify/, acedido 19,12,2021}

\bibitem{ranking}
\textit{https://revistamarieclaire.globo.com/Cultura/noticia/2021/12/spotify-divulga-rankings-com-artistas-e-musicas-mais-ouvidas-em-2021.html, acedido 19,12,2021}

\bibitem{paises}
\textit{https://www.jornaldenegocios.pt/empresas/tecnologias/detalhe/spotify-vai-passar-a-estar-presente-em-mais-80-paises, acedido 19,12,2021}

\bibitem{podcasts}
\textit{https://comunidadeculturaearte.com/spotify-revela-artistas-e-podcasts-mais-ouvidos-em-portugal-e-no-mundo-em-2021/
 acedido 19,12,2021}

\end{thebibliography}



\end{document}

